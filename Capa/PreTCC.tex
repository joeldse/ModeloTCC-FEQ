
\begin{titlepage}
\begin{center}

\begin{minipage}{0.15\linewidth}
    \begin{flushleft}
        \begin{figure}[H]
    		\includegraphics[scale=0.2]{Imagens/LOGO-UNICAMP.png}
    	\end{figure}
    \end{flushleft}
\end{minipage}
\begin{minipage}{0.6\linewidth}
	\begin{center}
		
		Universidade Estadual de Campinas
		
		Faculdade de Engenharia Química
		
		EQ950 x – Introdução ao Trabalho de Conclusão de Curso
		
	\end{center}
\end{minipage}
\begin{minipage}{0.22\linewidth}
    \begin{flushleft}
        \begin{figure}[H]
    		\includegraphics[scale=0.3]{Imagens/LOGO-EQ.jpg}
    	\end{figure}
    \end{flushleft}
\end{minipage}

\vfill





{\fontsize{28}{\baselineskip}\selectfont
Proposta de Trabalho de Conclusão de Curso 
}

\vfill





{\fontsize{28}{\baselineskip}\selectfont
1ª ou 2ª Etapa de avaliação
}

\vfill





{\fontsize{36}{\baselineskip}\selectfont
‘‘Título do Trabalho ’’
}

\vfill





\begin{minipage}{0.7\linewidth}
	{\onehalfspacing\fontsize{20}{\baselineskip}\selectfont
	
    Joel da Silva Evangelista
    
    Joel da Silva Evangelista 
    
    }
\end{minipage}
\begin{minipage}{0.25\linewidth}
	{\onehalfspacing\fontsize{18}{\baselineskip}\selectfont
	RA: 176463
	
	RA: 176463
	
	}
\end{minipage}

\vspace{10mm}






\begin{flushleft}
    {\fontsize{12}{\baselineskip}\selectfont
    Esta proposta de Trabalho de Conclusão de Curso, uma vez aprovada, será conduzida sob orientação do Prof. <nome completo do professor>, uma vez que se trata de tema desenvolvido em Programa Integrado de Formação (PIF)
    }
\end{flushleft}

\vfill





{\fontsize{14}{\baselineskip}\selectfont
	Campinas - SP

	\today
}

\end{center}
\end{titlepage}